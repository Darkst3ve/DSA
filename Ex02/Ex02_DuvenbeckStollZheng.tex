% LaTeX Template für Abgaben an der Universität Stuttgart
% Autor: Sandro Speth
% Bei Fragen: Sandro.Speth@studi.informatik.uni-stuttgart.de
%-----------------------------------------------------------
% Hauptmodul des Templates: Hier können andere Dateien eingebunden werden
% oder Inhalte direkt rein geschrieben werden.
% Kompiliere dieses Modul um eine PDF zu erzeugen.

% Dokumentenart. Ersetze 12pt, falls die Schriftgröße anzupassen ist.
\documentclass[12pt]{scrartcl}
% Einbinden der Pakete, des Headers und der Formatierung.
% Mit den \include und \input Befehlen können Dateien eingebunden werden:
% \include: Fügt einen Seitenumbruch nach dem Text ein
% \input: Fügt KEINEN Seitenumbruch nach dem Text ein
\input{../Usepackage.tex}
% LaTeX Template für Abgaben an der Universität Stuttgart
% Autor: Sandro Speth
% Bei Fragen: Sandro.Speth@studi.informatik.uni-stuttgart.de
%-----------------------------------------------------------
% Modul beinhaltet Befehl fuer Aufgabennummerierung,
% sowie die Header Informationen.

% Überschreibt enumerate Befehl, sodass 1. Ebene Items mit
\renewcommand{\theenumi}{(\alph{enumi})}
% (a), (b), etc. nummeriert werden.
\renewcommand{\labelenumi}{\text{\theenumi}}

% Counter für das Blatt und die Aufgabennummer.
% Ersetze die Nummer des Übungsblattes und die Nummer der Aufgabe
% den Anforderungen entsprechend.
% Gesetz werden die counter in der hauptdatei, damit siese hier nicht jedes mal verändert werden muss
% Beachte:
% \setcounter{countername}{number}: Legt den Wert des Counters fest
% \stepcounter{countername}: Erhöht den Wert des Counters um 1.
\newcounter{sheetnr}
\newcounter{exnum}

% Befehl für die Aufgabentitel
\newcommand{\exercise}[1]{\section*{Aufgabe \theexnum\stepcounter{exnum}: #1}} % Befehl für Aufgabentitel


\definecolor{javared}{rgb}{0.6,0,0} % for strings
\definecolor{javagreen}{rgb}{0.25,0.5,0.35} % comments
\definecolor{javapurple}{rgb}{0.5,0,0.35} % keywords
\definecolor{javadocblue}{rgb}{0.25,0.35,0.75} % javadoc
\lstdefinestyle{Java}
{
    language=Java,
	basicstyle=\ttfamily,
	keywordstyle=\color{javapurple}\bfseries,
	stringstyle=\color{javared},
	commentstyle=\color{javagreen},
	morecomment=[s][\color{javadocblue}]{/**}{*/},
	numbers=left,
	numberstyle=\tiny\color{black},
	stepnumber=2,
	numbersep=10pt,
	tabsize=4,
	showspaces=false,
	showstringspaces=false
}

\lstdefinestyle{CMD}
{
    basicstyle=\scriptsize\color{black}\ttfamily
}

% Formatierung der Kopfzeile
% \ohead: Setzt rechten Teil der Kopfzeile mit
% Namen und Matrikelnummern aller Bearbeiter
\ohead{Lennart Duvenbeck (2836913)\\
Timo Stoll (2976666)\\
David Zheng (3334362)}
% \chead{} kann mittleren Kopfzeilen Teil sezten
% \ihead: Setzt linken Teil der Kopfzeile mit
% Modulnamen, Semester und Übungsblattnummer
\ihead{Datenstrukturen und Algorithmen\\
Sommersemester 2018\\
Übungsblatt \thesheetnr}


% Counter für das Blatt und die Aufgabennummer.
% Ersetze die Nummer des Übungsblattes und die Nummer der Aufgabe
% den Anforderungen entsprechend.
% Definiert werden die Counter in FormatAndHeader.tex
% Beachte:
% \setcounter{countername}{number}: Legt den Wert des Counters fest
% \stepcounter{countername}: Erhöht den Wert des Counters um 1.
\setcounter{sheetnr}{2} % Nummer des Übungsblattes
\setcounter{exnum}{1} % Nummer der Aufgabe

% Beginn des eigentlichen Dokuments
\begin{document}
% Nutze den \exercise{Aufgabenname} Befehl, um eine neue Aufgabe zu beginnen.
% Möchtest du eine Aufgabe in der Nummerierung überspringen, schreibe vor der Aufgabe: \stepcounter{exnum}
% Möchtest du die Nummer einer Aufgabe auf eine beliebige Zahl x setzen, schreibe vor der Aufgabe: \setcounter{exnum}{x}

% Aufgabe 1
\exercise{}
\begin{figure}[h]
\begin{center}
\includegraphics[scale=0.5]{complexity.png}
\caption{Auswertung der vorgegebenen Funktionen in halb-logarithmischer Darstellung}
\end{center}
\end{figure}
Angabe der Reihenfolge der Funktionen nach ihrer asymptotischen Komplexität für $n \rightarrow \infty $\\
\begin{center}
\begin{tabular}{l r}
$f_3=O(n!)$ & Faktoriell\\
$f_5=O(2^n)$ & Exponentiell\\
$f_6=O(n^3)$ & Polynomiell\\
$f_2=O(n^2 \log(n))$ & Polynomiell\\
$f_1=O(n)$ & Polynomiell\\
$f_4=O(\log(n))$ & Logarithmisch\\
\end{tabular}
\end{center}


% Aufgabe 2
\exercise{}
Lösung im beigefügten Eclipse-Projekt
\newpage

% Aufgabe 3
\exercise{}
\renewcommand{\labelenumii}{\alph{enumii}}
\begin{enumerate}
\item{Die Schleife über i geht von $2 \cdot n$ bis Null wobei bei jeder Iteration der Wert von i um Eins reduziert wird. Die Komplexität dieser Schleife beträgt also $2n$. Die Schleife über j geht von 0 bis j wobei bei jeder Iteration der Wert von j um Zwei erhöht wird. Die Komplexität dieser Schleife beträgt also $\frac{n}{2}$. Insgesamt also ist die asymptotische Komplexität $O(n)$.}

\item{Die Gesamtkomplexität von verschachtelten Schleifen ist das Produkt der einzelnen Komplexitäten. Die Schleife über i hat Komplexität $n$, die Schleife über j hat $\frac{n}{2}$ und die Schleife über k hat Komplexität $\frac{n}{6}$ (Wenn der Index der Schleife in jeder Iteration mit einem Wert addiert wird ist die Komplexität der Schleife die Differenz von Start- und Endwert geteilt durch die \glqq Schrittlänge\grqq). Die asymptotische Komplexität des Algorithmus beträgt also $O(n^3)$.}

\item{Der Algorithmus ruft sich selbst rekursiv insgesamt n mal auf. Bei jeder Rekursion entstehen zwei Funktionsaufrufe. Also ist die asymptotische Komplexität $O(2^n)$. Es ist zu bemerken dass keinerlei Rechenoperationen durchgeführt werden, am Ende der Rekursion steht lediglich die Ausgabe von n.}

\item{Nach Erreichen von $\frac{n}{2}$ wird die Schleife abgebrochen. Bis dahin wird in jeder Schleifeniteration mit Zwei multipliziert. Die Komplexität ist also die Lösung der Gleichung $2^x=\frac{n}{2}$ also $O(\log{n})$.}

\item{Die Komplexität ist das Produkt der Komplexitäten der beiden Schleifen. Dass $n$ durch $\frac{n}{2}$ ersetzt wird ist dabei für die Abschätzung der Komplexität irrelevant. Die Komplexität der Schleife über i ist $n$ und die der Schleife über j $\log{n}$ da in jeder Iteration j verdoppelt wird. Insgesamt also ist die asymptotische Komplexität $O(n\log{n})$.}

\item{Nur die Schleife über i hängt in ihrer Komplexität von n ab. Die Komplexität der Schleife über j ist bezüglich n konstant. Also ist die Gesamtkomplexität $O(n)$.}
\end{enumerate}

% Ende des Dokuments
\end{document}

