% LaTeX Template für Abgaben an der Universität Stuttgart
% Autor: Sandro Speth
% Bei Fragen: Sandro.Speth@studi.informatik.uni-stuttgart.de
%-----------------------------------------------------------
% Hauptmodul des Templates: Hier können andere Dateien eingebunden werden
% oder Inhalte direkt rein geschrieben werden.
% Kompiliere dieses Modul um eine PDF zu erzeugen.

% Dokumentenart. Ersetze 12pt, falls die Schriftgröße anzupassen ist.
\documentclass[12pt]{scrartcl}
% Einbinden der Pakete, des Headers und der Formatierung.
% Mit den \include und \input Befehlen können Dateien eingebunden werden:
% \include: Fügt einen Seitenumbruch nach dem Text ein
% \input: Fügt KEINEN Seitenumbruch nach dem Text ein
\input{../Usepackage.tex}
% LaTeX Template für Abgaben an der Universität Stuttgart
% Autor: Sandro Speth
% Bei Fragen: Sandro.Speth@studi.informatik.uni-stuttgart.de
%-----------------------------------------------------------
% Modul beinhaltet Befehl fuer Aufgabennummerierung,
% sowie die Header Informationen.

% Überschreibt enumerate Befehl, sodass 1. Ebene Items mit
\renewcommand{\theenumi}{(\alph{enumi})}
% (a), (b), etc. nummeriert werden.
\renewcommand{\labelenumi}{\text{\theenumi}}

% Counter für das Blatt und die Aufgabennummer.
% Ersetze die Nummer des Übungsblattes und die Nummer der Aufgabe
% den Anforderungen entsprechend.
% Gesetz werden die counter in der hauptdatei, damit siese hier nicht jedes mal verändert werden muss
% Beachte:
% \setcounter{countername}{number}: Legt den Wert des Counters fest
% \stepcounter{countername}: Erhöht den Wert des Counters um 1.
\newcounter{sheetnr}
\newcounter{exnum}

% Befehl für die Aufgabentitel
\newcommand{\exercise}[1]{\section*{Aufgabe \theexnum\stepcounter{exnum}: #1}} % Befehl für Aufgabentitel


\definecolor{javared}{rgb}{0.6,0,0} % for strings
\definecolor{javagreen}{rgb}{0.25,0.5,0.35} % comments
\definecolor{javapurple}{rgb}{0.5,0,0.35} % keywords
\definecolor{javadocblue}{rgb}{0.25,0.35,0.75} % javadoc
\lstdefinestyle{Java}
{
    language=Java,
	basicstyle=\ttfamily,
	keywordstyle=\color{javapurple}\bfseries,
	stringstyle=\color{javared},
	commentstyle=\color{javagreen},
	morecomment=[s][\color{javadocblue}]{/**}{*/},
	numbers=left,
	numberstyle=\tiny\color{black},
	stepnumber=2,
	numbersep=10pt,
	tabsize=4,
	showspaces=false,
	showstringspaces=false
}

\lstdefinestyle{CMD}
{
    basicstyle=\scriptsize\color{black}\ttfamily
}

% Formatierung der Kopfzeile
% \ohead: Setzt rechten Teil der Kopfzeile mit
% Namen und Matrikelnummern aller Bearbeiter
\ohead{Lennart Duvenbeck (2836913)\\
Timo Stoll (2976666)\\
David Zheng (3334362)}
% \chead{} kann mittleren Kopfzeilen Teil sezten
% \ihead: Setzt linken Teil der Kopfzeile mit
% Modulnamen, Semester und Übungsblattnummer
\ihead{Datenstrukturen und Algorithmen\\
Sommersemester 2018\\
Übungsblatt \thesheetnr}


% Counter für das Blatt und die Aufgabennummer.
% Ersetze die Nummer des Übungsblattes und die Nummer der Aufgabe
% den Anforderungen entsprechend.
% Definiert werden die Counter in FormatAndHeader.tex
% Beachte:
% \setcounter{countername}{number}: Legt den Wert des Counters fest
% \stepcounter{countername}: Erhöht den Wert des Counters um 1.
\setcounter{sheetnr}{9} % Nummer des Übungsblattes
\setcounter{exnum}{1} % Nummer der Aufgabe

% Beginn des eigentlichen Dokuments
\begin{document}
% Nutze den \exercise{Aufgabenname} Befehl, um eine neue Aufgabe zu beginnen.
% Möchtest du eine Aufgabe in der Nummerierung überspringen, schreibe vor der Aufgabe: \stepcounter{exnum}
% Möchtest du die Nummer einer Aufgabe auf eine beliebige Zahl x setzen, schreibe vor der Aufgabe: \setcounter{exnum}{x}

% Aufgabe 1
\exercise{}

Teilaufgabe a:\\\\
\begin{tabular}{|c|c|c|c|c|c|c|c|c|c|c|}
\hline 
Muster & g & g & h & f & g & g & h & f & g & f\\ 
\hline 
Längster Präfix & 0 & 1 & 0 & 0 & 1 & 2 & 3 & 4 & 5 & 0\\ 
\hline 
\end{tabular}
\\\\
Teilaufgabe b:
\begin{table}[h!]
\begin{tabular}{|c|c|c|c|c|c|c|c|c|c|c|c|c|c|c|c|c|c|c|c|}
\hline 
Muster & g&g&h&f&h&g&g&h&g&g&h&f&g&g&h&f&g&g&h\\ 
\hline 
Index  & 1&2&3&4&5&6&7&8&9&10&11&12&13&14&15&16&17&18&19\\
\hline
\hline
Muster &f&g&h&g&g&h&f&g&g&h&f&g&f&f&g&f&h&g&h\\
\hline
Index & 20&21&22&23&24&25&26&27&28&29&30&31&32&33&34&35&36&37&38\\ 
\hline
\end{tabular}
\caption{Indices für $t_1$}
\end{table}\\
Suche ab 1: Mismatch bei 5. Längster Präfix hat Länge 0. Schiebe Suchposition nach 5.\\
Suche ab 5: Mismatch beim ersten Index. Schiebe Suchposition nach 6.\\
Suche ab 6: Mismatch bei 9. Längster Präfix hat Länge 0. Schiebe Suchposition nach 9.\\
Suche ab 9: Mismatch bei 18. Längster Präfix hat Länge 5. Schiebe Suchposition nach 13.\\
Suche ab 14: Mismatch bei 22. Längster Präfix hat Länge 5. Schiebe Suchposition nach 17.\\
Suche ab 17: Mismatch bei 22. Längster Präfix hat Länge 1. Schiebe Suchposition nach 21.\\
Suche ab 21: Mismatch bei 22. Längster Präfix hat Länge 0. Schiebe Suchposition nach 23.\\
Suche ab 22: String $p_1$ gefunden.

% Aufgabe 2
\exercise{}
TODO

% Aufgabe 3
\exercise{}
\begin{lstlisting}
^a...(a|b)$
^(Apfel|Kirsch|Birn|Obst)baum(plantage)?$
^Rebe(kk|k|ck|cc)a(h)?$
^(((X|Y|Z)a(b)?)|(Bla(blabla)+))ooh..$
\end{lstlisting}



% Ende des Dokuments
\end{document}

